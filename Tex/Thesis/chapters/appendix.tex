
\begin{appendices}

\renewcommand{\thesection}{\Alph{section}}

\section{Use of natural units (GeV)}

\label{App}

This section is based on \cite{rm}.

A fundamental equation of special relativity is, $$ E^2 = p^2c^2 + m^2c^4 $$ where $E$ is the energy of the particle, $p$ is its momentum, $m$ is the mass and $c$ is the speed of light. When a particle is at rest its momentum is 0, this gives us Einstein's mass-energy equivalence, $E = mc^2$.  Using the units GeV for Energy, GeV$/c$ for momentum and GeV$/c^2$ for mass we get the equivalence,  $$E^2 = p^2 + m^2 $$ The papers published in the ATLAS and CMS experiment use the notation GeV for mass, energy and momentum. We will follow the same convention.

The momentum $p$ of a particle is actually a 3-dimensional vector $\overrightarrow{p} = (p_{x}, p_{y}, p_{z})$ stating the particle's momentum in 3 directions in 3-d space. For a particle with non-zero mass the momentum of a particle is $\overrightarrow{p} = m\overrightarrow{v}$ where $\overrightarrow{v}$ is the 3-dimensional velocity and $m$ is the mass. The 4-momentum of a particle is defined as $(p_{x}, p_{y}, p_{z}, E)$. This defines the full kinematics of a particle as if we know the particle's momentum and energy we can compute its mass using the relation, $$ m = \sqrt{E^2 - p^2} $$

Similarly, if we know any two quantities out of momentum, mass and energy we can compute the third deterministically by equations of special relativity specified above. 

\clearpage

\end{appendices}